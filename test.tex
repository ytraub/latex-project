\documentclass[a4paper,12pt]{article}

% Pakete einbinden
\usepackage[utf8]{inputenc}
\usepackage[T1]{fontenc}
\usepackage{graphicx}
\usepackage{hyperref}
\usepackage{makeidx}
\usepackage{xcolor}
\usepackage{amsmath}
\usepackage{listings}
\usepackage{qrcode}
\usepackage{tikz}
\makeindex

% Dokumentbeginn
\begin{document}

% Titelseite
\title{Erweitertes LATEX-Dokument}
\author{Max Mustermann, Erika Musterfrau}
\date{\today}
\maketitle

% Abstract
\begin{abstract}
Dieses Dokument demonstriert die Nutzung von LATEX mit fortgeschrittenen Features wie Schriftformatierungen, mathematischen Formeln, Verweisen, QR-Codes, Grafiken, Tabellen und mehr.
\end{abstract}

% Inhaltsverzeichnis
\tableofcontents
\newpage

% Liste der Abbildungen
\listoffigures
\newpage

% Hauptteil (Beispielinhalt)
\section{Einleitung}
Dies ist die Einleitung. Hier wird das Thema kurz beschrieben.

\section{Schriftformatierungen}
\textbf{Fettdruck}, \textit{Kursivdruck}, und \textcolor{blue}{blaue Schriftfarbe} können einfach mit LATEX umgesetzt werden. Die Schriftgröße hier ist \Large manuell angepasst.

\section{Mathematische Formeln}
Unnummerierte Formel:
\[
a^2 + b^2 = c^2
\]
Nummerierte Formel:
\begin{equation}
\int_{0}^{\infty} e^{-x^2} dx = \frac{\sqrt{\pi}}{2}
\label{eq:gauss}
\end{equation}

Verweis auf die nummerierte Formel \eqref{eq:gauss} im Text.

\section{Verweise}
Dieser Abschnitt enthält einen Verweis auf \hyperref[sec:tabellen]{Abschnitt \ref*{sec:tabellen}} auf Seite \pageref{sec:tabellen}.

\section{Externe Links}
Ein Link zu \href{https://www.latex-project.org}{LATEX-Projektseite}.

\section{QR-Code}
Ein QR-Code zur LATEX-Projektseite:
\qrcode{https://www.latex-project.org}

\section{Selbsterstellte Grafiken}
Eine einfache Grafik:
\begin{tikzpicture}
\draw[thick,->] (0,0) -- (2,0) node[right]{x};
\draw[thick,->] (0,0) -- (0,2) node[above]{y};
\draw[thick] (0,0) -- (1,1);
\end{tikzpicture}

\section{Bilder}
Ein eingebundenes Bild:
\begin{figure}[h!]
\centering
\includegraphics[width=0.5\textwidth]{example-image}
\caption{Eine Beispielabbildung}
\label{fig:beispiel}
\end{figure}

\section{Codelisting}
Ein Beispielcode:
\begin{lstlisting}[language=Python, caption={Beispielcode in Python}]
def hello_world():
    print("Hello, world!")
\end{lstlisting}

\section{Tabellen}
\label{sec:tabellen}
Eine Beispielstabelle:
\begin{table}[h!]
\centering
\begin{tabular}{|c|c|c|}
\hline
A & B & C \\
\hline
1 & 2 & 3 \\
\hline
4 & 5 & 6 \\
\hline
\end{tabular}
\caption{Eine einfache Tabelle}
\end{table}

\section{Fussnoten}
Ein Beispiel für eine Fussnote\footnote{Dies ist eine Fussnote.} im Text.

% Bibliographie
\newpage
\bibliographystyle{plain}
\bibliography{literatur}

% Index
\newpage
\printindex

\end{document}
